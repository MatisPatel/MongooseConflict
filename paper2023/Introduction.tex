\section{Introduction}

Humans have a huge impact on the structure of the natural world. The impact these changes will have on the ecosystems of the world is not always clear but is are largely considered to be catastrophic. 
One of the key effects of global warming and human disruption is an increase in extreme states. Anthropogenic global warming drives high maximum temperatures and incredibly low minimum temperatures. Equally, human exploitation concentrates resources such as food refuse to areas of human habitation while converting nearby wilderness into human specific crop monocultures.  
Behavioural responses in animals are extremely important as they represent some of the most immediate observable effects of an environmental perturbation. These behavioural responses can lead to short and long term success or can trap species into suboptimal ecological and evolutionary patterns that can lead to extinction or a reduction in resilience to future changes.

Some of the most well studied behavioural changes have been in urban environments. Many bird species have adapted their song frequencies to avoid interference from urban soundscapes and the acoustic differences between a forest and a maze of buildings \parencite{slabbekoornEcologyBirdsSing2003}. 
More recently research has shown urban foxes are bolder than rural foxes at investigating unfamiliar food opportunities \parencite{mortonUrbanFoxesAre2023}. This is possibly due to better familiarity with a wider range of food opportunities in urban environments.

Behavioural changes are immediate responses from animals to variation in the environment and are fundamentally built on responses that have evolved to respond to expected natural variation. This means they are not necessarily advantageous long term and for many animals the constant and accelerating change that humans create is very much an out of context problem. For example, African Wild Dogs on hot days reduce daytime activity and switch to foraging at night more. However, they restrict this switch to moonlit nights and when raising pups reduce this nighttime activity even further \parencite{rabaiottiCopingClimateChange2019}. As the planet heats this compensation will eventually be inadequate, and the population growth will suffer. The Wild Dogs are executing a behavioural response evolved to deal with transient heatwaves in the face of an ever-increasing ratcheting of global temperatures. 

Research into behavioural response has largely focused on responses to abiotic factors or alien species \parencite{wongBehavioralResponsesChanging2015}. However, for many animal species especially those in harsher environments social behaviours within cooperative groups can have large fitness effects \parencite{downingGroupFormationEvolutionary2020}. How these social behaviours interact with changes in the environment is not always clear and depends on and interplay of within and between group interactions. 

We model the between and within group social behaviours of a popualtion based on the Banded Mongoose (\textit{Mungos mungo}). They represent a useful study system as they exhibit within group cooperation and between group conflict with large fitness effect (QUOTE numbers). They also inhabit a semi-arid region with year-to-year variation in resource abundance. In drought years they experience harsh conditions whereas in high rainfall years they have bountiful feeding opportunities. In response, they have likely evolved a sophisticated conditional behavioural responses to modulate their cooperation and conflict. By modelling the system we show how the mongooses might change their within group cooperation and between conflict in response to resource availability. We also then highlight how these conditional rules are violated when large scale environmental changes shift the distribution of expected states. 

((Actual summary of results here))

These results provide testable predictions for changes in observed conflict and cooperation within wild populations. The method we use could also be easily applied to other species ad systems and provides a general framework for assessing the behavioural fragility of a species to changes in the environment. 
