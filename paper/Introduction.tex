\section{Introduction}\todo{this whole section needs citations}

Intergroup conflict is thought to be the key driver in the evolution of cooperation. However our understanding of the key evolutionary and ecological drivers of inter group conflict is far from complete. In social mammals both between and within group conflict is variable and seems positively or negatively correlated with between and within group cooperation. This variety of responses shows that a better understanding of the driver of cooperation and conflict are needed to fully explain how within and between group conflict evolves in social mammals. \todo{this needs to be be better}

Previous models have shown that intergroup conflict can favour within group cooperation. However these models often link the payoffs of cooperation and conflict so as to enable a direct synergism between the two. This approach may make sense for some complicated behaviours such as tribal warfare in human societies, but simpler behaviours or those more prevalent in social mammals more generally may not follow this pattern. Specifically, we might expect the opposite to be true --- performing well in intergroup encounters makes one less willing or able to cooperate with others or vice versa.

Previous work has also focused on non-conditional traits that do not take into account the fact that benefits and costs of intergroup conflict are highly dependent on the internal state of the group and the environment. The economic and political theory literature on war argues that resource inequality drives human conflict but few studies have explored that same logic to the evolution of intergroup conflict in social mammals. 

In this paper we explicitly model a group structured population where each group varies in number of individuals and in resource quality. Groups engage in Tullock contests where the winner gains one of the losers resources. Individuals play conditional strategies of intergroup conflict and cooperation and they are allowed perfect knowledge of their own state but not their opponents. \todo{findings summarised here}