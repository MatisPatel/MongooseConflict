\section{Introduction}

Intergroup conflict is thought to be the key driver in the evolution of cooperation \citep{radfordWithingroupBehavioural2016,kappelerMindGap2010,barkerWithingroupCompetition2012}. 
However our understanding of the key evolutionary and ecological drivers of inter group conflict is far from complete. 
In social organisms both between and within group conflict is variable and seems positively or negatively correlated with between and within group cooperation \citep{radfordWithingroupBehavioural2016}. 
This variety of responses shows that a better understanding of the driver of cooperation and conflict are needed to fully explain how within and between group conflict evolves. 
These traits of cooperation and conflict can have large fitness consequences \citep{thompsonCausesConsequences2017,vitikainenLiveLong2019}.

Previous models have shown that intergroup conflict can favour within group cooperation. 
However, these models often link the payoffs of cooperation and conflict so as to enable a direct synergism between the two \citep{choiCoevolutionParochial2007,lehmannWarEvolution2008a}. 
This assumption is suited to answer certain questions in the evolution of human societies however hides the tension between cooperation and conflict that exists in other cooperative groups. 
Specifically, we might expect performing well in intergroup encounters makes one less willing or able to cooperate with others or vice versa.  \todo{example of where it does and doesnt work}

In nature organisms evolve complex behaviours that respond adaptively to the current situation it finds itself in. Individuals in smaller groups might fight harder whereas in larger groups they might cooperate more as benefits are synergistic. These state dependent behaviours are crucial are crucial as they allow conditional behaviour that is adapted to the individuals specific circumstance rather than only doing what is optimal on average. 

\Citet{kropotkinMutualAid1902} proposed that cooperation increased in harsh environments. Evidence since then has been varied, it is true that harsh environments are colonised at a greater rate by cooperative species \citep{cornwallisCooperationFacilitates2017}. However, it is not clear if harshness itself selects for cooperation. Previous theoretical work has shown that resource limitation does select for cooperative strategies by essentially modifying the payoffs of the underlying game \citep{requejoEvolutionCooperation2011,smaldinoIncreasedCosts2013}. Establishing how cooperation and conflict interact with resources however has been hard as previous models have not explicitly separated the sociality of between-group conflict and within-group cooperation from each other and with an underlying ecological measure of harshness. 

In this paper we model two separate traits one controlling between-group conflict and one controlling within-group cooperation. They are both social traits that are fully conditional on the individuals group's size and resources. This removes the correlations between within group and between group behaviour, of \citet{choiCoevolutionParochial2007} and \citet{lehmannWarEvolution2008}, of previous models and explicitly allows the payoffs of these behaviours to be mediated by explicit modelling of an underlying resource that determines individual fitness. This allows us to better understand the trade-offs between within-group and between-group behaviours interacting with changing ecological conditions.

We investigate how cooperation and conflict evolve when resources are scarce or abundant. We also analyse how the wealth and size of a group affects their investment into cooperation and conflict. We find that harsher environments increase cooperation but only increase conflict up to a point beyond which it decreases again. We also find that in harsh environments conflict increases with richness but in benign environments conflict decreases with quality. Group size increasing decreases both cooperation and conflict in all environments. 