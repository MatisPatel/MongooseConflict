\section{Discussion}

We found that as the environment became more harsh, in that resources disappeared at a greater rate then they were generated, then conflict increased up to a maximum value after which conflict decreased again. The higher the rate of conflicts the more investment into conflict was made by individuals however this did saturate at high encounter rates and no further increase in conflict was observed. 
For cooperation we found that increasing encounter rate and environmental harshness both had a small positive effect on cooperation. 

We found that investment into conflict increased with wealth in harsh environments and decreased with wealth in benign environments. And we found that individual investment into both cooperation and conflict decreased with group size but group investment in cooperation still increased overall with group size. 

\Citet{kropotkinMutualAid1902} proposed that cooperation increased in harsh environments. Evidence since then has been varied, it is true that harsh environments are colonised at a greater rate by cooperative species \citep{cornwallisCooperationFacilitates2017}. However, it is not clear if harshness itself selects for cooperation. Previous theoretical work has shown that resource limitation does select for cooperative strategies by essentially modifying the payoffs of the underlying game \citep{requejoEvolutionCooperation2011,smaldinoIncreasedCosts2013b}. Our results also show an increase in cooperation from benign to poor environments. However, the increase in within group cooperation leads to higher levels of between group conflict. So cooperation does increase but only towards group members. Which leads to more harmonious groups united in conflict rather than any type of utopian population wide cooperation.

The shift in state distributions drives the pattern we see in how conflict varies with group wealth and with harshness. In harsh environments all groups are poor which leads to low conflict as encountered groups have no resources to steal and equally in benign environments the need to fight is low amongst naturally rich groups. This maximisation of conflict in intermediate environments is relevant when thinking about environmental change. Supplementary feeding is performed in a number of conservation strategies primarily in scavenging species and predators \citep{oroTestingGoodness2008}. Negative impacts such as stress and disease spread have been analysed before \citep{murrayWildlifeHealth2016}. Our results point to the possibility of a more indirect result in that increasing feeding for a species in harsh conditions might drive selection for higher levels of aggression especially if the feeders are claimable or in some way controllable by a group. In addition, already well provisioned species if their supplementation is removed or disrupted could also increase their levels of conflict. These predictions are however evolutionary ones. On shorter timescales it would be useful to extend our results by displacing evolved strategies into new environmental regimes and measuring changes in expressed cooperation and conflict. 

The effect of low and high harshness in our model is partially driven by the fact we do not allow groups at the maximum resource level to take resources from other groups. We could instead allow rich groups to essentially swap a claimed resource for an old resource generating an empty patch with the discarded resource and removing a resource from the losing group. This would remove some of the selection against conflict at high resource levels as fights between two groups at the maximum resource levels would still harm the loser. This would probably have the effect of raising the overall investment in conflict amongst rich groups but would not affect the overall pattern of intermediate harshness maximising conflict as the same non-interaction still makes sense for poor groups and intermediate harshness will still maximise the number of possible fights occurring in a population as the distribution of states is more even. 

It is know that the major cost from intergroup conflict in some species is in mortality from the fight \cite{cantBandedMongooses2016}. However, our model does not include direct encounter based mortality as a cost of fighting. To include this we would need to include a more sophisticated fight logic that allowed avoidance and initiation of a fight. Otherwise mortality would just become a constant scaling cost from encounter rate and not a cost of investment into conflict. Also if mortality is dependent or independent of personal investment could play an important role. To maintain comprehension we did not explore these angles in the basic model but armed with the understanding from the paper future work is well placed to answer these questions.

\todo{concluding para}

\begin{enumerate}
    \item Harsher environs favoured increased conflicy up until too much of te popualtion was poor.
    \item Harsher environs always selected for higher cooperation though change was small relatively 
    \item encounter rate increasing drove higher conflict in harsh environemnts but again with an intermediate optimum. 
    \item cooperation increased with higher encounter rates but the shift was again quite small.
    \item in harsh environs rich groups invested most in conflict in benign environs poor groups invested most. 
    \item increasing group size decreased both conflict and cooperation however group level cooperation went up and conflict went down. 
\end{enumerate}

\subsection{comparisons} 
\begin{enumerate}
    \item yes groups more social in harsh environs but they are investing in between group conflict rather than within group cooperation (though obviously structural)
    \item "altruism" is higher in harsh environs.
    \item larger groups are more cooperative but each individual is less (economies of scale + private benefit).
    \item larger groups invest less in fighting (resources worth less? relatedness decreases and no private benefit to fighting)
\end{enumerate}

\subsection{next steps}
\begin{enumerate}
    \item add mortality from fights to give personal cost more directly. currently asymmetric benefits but not costs.
    \item allow resources to be destroyed and groups to still deny resources even when they can't gain them.
    \item stop fights with empty patches jsut have promotion for dipspersal.
    \item look at perturbations and assess out of context response. 
\end{enumerate}